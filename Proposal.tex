\RequirePackage[l2tabu, orthodox]{nag}
\documentclass[a4paper]{scrartcl}
\usepackage{xltxtra} 		% loads fixltx2e, metalogo, xunicode, fontspec
\usepackage{microtype}      	% typography pedantry
\usepackage[british]{babel}
\usepackage{ifthen}
\usepackage{color}
\usepackage{amsmath}
\usepackage{amsthm}
\usepackage{thmtools}
\usepackage{paralist}
\usepackage{nameref}
\usepackage{cleveref}
\usepackage{csquotes}
\usepackage{fancyvrb}
\usepackage{unicode-math}
\usepackage[autocite=footnote,citestyle=authoryear-comp,bibstyle=authoryear,dashed=false,isbn=false,doi=false,url=false,backend=biber]{biblatex}
\usepackage[bookmarks,hidelinks]{hyperref}

\setcounter{secnumdepth}{0}

%%% Custom LaTeX preamble
% Linux Libertine (free, wide coverage, not only for Linux)
\setmainfont{Kp}
\setsansfont{Linux Biolinum O}
%\setsansfont[
   %BoldFont={Myriad Pro Bold},
   %ItalicFont={Myriad Pro Italic},
   %BoldItalicFont={Myriad Pro Bold Italic}
   %]{Myriad Pro}
\setmonofont[HyphenChar=None,Scale=MatchLowercase]{DejaVu Sans Mono}
\setmathfont{latinmodern-math.otf}

%%% User specified packages and stylesheets
\DefineVerbatimEnvironment%
   {code}{Verbatim}
   {} % options

% hyperlinks
\urlstyle{same} % normal text font (alternatives: tt, rm, sf)
\hypersetup{%
  pdftitle={Big operators in Agda},
  pdfauthor={Leonhard Markert (lm510), Emmanuel College}
}

% KOMA-script styles
%\addtokomafont{disposition}{\rmfamily}

% bibliography
\addbibresource{Bibliography.bib}

%%% Title Data
\title{Big operators in Agda}
\author{Part III project proposal\\
	Leonhard Markert, Emmanuel College}
\date{\textit{Supervisors:} Timothy Griffin and Dominic Mulligan}
%\date{Proposed Supervisors:\\Timothy Griffin and Dominic Mulligan}

%\bibliographystyle{alpha}

%%% Body
\begin{document}

\maketitle

\begin{abstract}
\enquote{Big operator} notation is common in algebraic reasoning.
%It is a useful tool to express formulae in a succinct and abstract way.
A formal treatment must ensure that an operator that is lifted into a big operator satisfies a set of laws.
Big operators have been implemented as part of an extension of Coq, a dependent type theory.
Agda, a more recent dependent type theory, currently lacks such a library.

I propose to implement a library for reasoning uniformly about big operators in Agda.
Since Agda and Coq differ in idioms, constructs and syntax, I will consult the existing big operators library for high-level ideas, but the proposed library will be designed from scratch.
\end{abstract}

\section{Project idea}

In many areas of mathematics, \enquote{big operators} are a commonly used and versatile tool which provides a natural way to express an operation iterated over some set. Making big operators available in a formal theorem proving environment like Agda therefore allows for a natural style of algebraic reasoning.

\subsection{Background}

%Big operators are pervasive in many areas of mathematics.
A big operator is constructed from a small operator, a collection of objects to iterate over, and an expression containing a free variable that can be bound to objects from the collection. For example, \(\sum_{x \in X} e(x)\) can be constructed from the small operator \(+\) along with some collection \(X\) and an expression \(e\).

This assumes the existence of a notion of \enquote{collection} that can be iterated over. One way to formalise such collections is as an enumerable finite set.

As a concrete example for how a big operators library can be used, consider a few steps from a proof that matrix multiplication is associative:

\begin{align*}
&\left(A \left(B C\right)\right)_{i,j} \\
&= \dots \\
&= \sum_k \sum_l  A_{i,k} \cdot \left( B_{k,l} \cdot C_{l,j} \right) \\
&= \sum_l \sum_k  A_{i,k} \cdot \left( B_{k,l} \cdot C_{l,j} \right)
  && \text{by commutativity and associativity of \(+\)} \\
&= \sum_l \sum_k \left( A_{i,k} \cdot B_{k,l} \right) \cdot C_{l,j}
  && \text{by associativity of \(\cdot\)} \\
&= \sum_l \left( \sum_k A_{i,k} \cdot B_{k,l} \right) \cdot C_{l,j}
  && \text{by right-distributivity of \(\cdot\) over \(+\)} \\
&= \dots \\
&= \left( \left( A  B \right) C\right)_{i,j}
\end{align*}

This example proof requires swapping the order of big operators and moving factors into or out of a big operator expression. Such lemmas, along with their necessary conditions, need to be formalised before one can reason with big operators in a theorem prover.

\subsection{Motivation}

The question of how to formalise generalised shortest-path problems in Agda provided the initial motivation for this project.
Some work has been done in this area using \emph{Coq}, an interactive theorem prover.\footnote{A formalisation in Coq of the correctness proof for a generalised version of Dijkstra's algorithm presented in \autocite{sobrinho_routing_2010} can be found here: \url{http://www.cl.cam.ac.uk/~tgg22/metarouting/rie-1.0.v}. Coq can be obtained from its website, \url{http://coq.inria.fr}. For an introduction to Coq, see \autocite{bertot_interactive_2004}.}

Fundamentally, \emph{Agda} is both a dependently typed programming language and a higher-order type theory, as is Coq.\footnote{Both represent propositions as types and proofs as terms. However, the underlying type theories as well as the concrete implementations of the two proof assistants differ in many ways. Coq's type system is based on Coquand and Huet's \enquote{calculus of inductive constructions}---see \textcite{bertot_interactive_2004} for an overview---whereas Agda's is based on Luo's \enquote{unifying theory of dependent types} \autocite{luo_computation_1994}. A good introduction to the relationship between propositions and types, proofs and programs is \textcite{sorensen_lectures_2006}. Agda can be obtained from \url{http://wiki.portal.chalmers.se/agda/pmwiki.php}.}


One important difference between Agda and Coq is the use of proof-constructing metaprogramming facilities called \emph{tactics} in idiomatic Coq code.
Tactics make proofs easier to write at the expense of hiding the \enquote{real} proof of a statement from the user.
%Whilst the underlying proof constructed by the tactic-metaprogram is of course still correct, as guaranteed by Coq's typechecker, it is hard to see \emph{why} this is the case just by reading a proof script.
Agda does not support tactics, and therefore proof terms must be provided by the user via a process of refinement. This is aided by Agda's flexible implementation of dependent pattern matching.
%There is no obfuscating metaprogramming layer sitting between the user and the \enquote{real} proof, making it easier to follow.

It soon became clear that a lot of the infrastructure required to write proofs about shortest-path algorithms does not yet exist in Agda.
One critical dependency of the aforementioned proof is a module that formalises the notion of big operators in Coq---no such library currently exists in Agda, even though it would be useful in a large range of contexts.\footnote{The \texttt{bigop} module was originally part of Ssreflect, an extension of Coq \autocite{gonthier_introduction_2010}. In version 1.5 it was moved to the Mathematical Components library. Both are available at \url{http://ssr.msr-inria.inria.fr/}. \textcite{bertot_canonical_2008} describes the design of this module.}

\subsection{Project goal}

The goal of this project is to build up a set of \emph{reusable} libraries that allow proving the theorem \enquote{any square matrix over a semiring is a semiring} in Agda in a way that reads naturally even to someone not used to formalising mathematics in a theorem prover.

This will depend on a implementation of big operators in Agda, which in turn depends on a notion of enumerable finite sets. A library formalising basic linear algebra will be required as well. To my best knowledge, neither exists at this time.

Formalisations of big operators and linear algebra have been written in Coq.
I expect that most of the fundamental notions and mechanisms carry over from Coq to Agda, but due to the differences between the two languages described in the Motivation section, the concrete implementation will have to be redesigned from scratch, as porting modules directly from Coq to Agda will not result in readable code.

Finite sets, linear algebra and big operators are useful in many contexts, so it is reasonable to expect that libraries formalising them will be of general interest to the wider Agda community. I will therefore design any auxiliary libraries with reusability in mind, aiming to write idiomatic code that could be included in the Agda standard library.

\section{Workplan}
\label{workplan}

\subsection{Michaelmaes term until Christmas}

In order to design and implement the proposed library in Agda, I will have to become proficient in reading and writing Agda code. This includes learning how program with dependent types and put them to good use, and how code is structured and written idiomatically in Agda. I will set aside some time during the last weeks of Michaelmas term to write simple programs and proofs in Agda. From the end of Michaelmas until Christmas, I will keep reading and writing Agda code and learn more about its type system, the module system and common idioms.

I will also have to acquire a good understanding of the mathematics of big operators. This involves finding answers to questions such as, \enquote{what are the minimum requirements for a small operator and a carrier set so a big operator can automatically be derived from it?}, and exploring what additional rules governing big operator can be derived from properties of the underlying algebra.\footnote{Gondran and Minoux' book \emph{Graphs, Dioids and Semirings} will provide much of the required background on algebra \autocite{gondran_graphs_2008}.} I have already begun to review previous work on the mathematics and implementation of big operators.\footnote{A library formalising big operators as folds has recently been developed. It differs from the proposed project in its intended use: I aim to implement a module for \emph{reasoning} about big operators, whereas the existing library makes it possible to \emph{program} with big operators \autocite{gustafsson_foldable_2014}.} During the last weeks of Michaelmas term I will find and read more literature about the relationship between big operators and abstract algebra. The time between the end of term and Christmas will be allocated to exploring this relationship further, and to find ways of encoding them in Agda.

\subsection{Lent term}

Given my limited exposure to Agda and its module system so far, the design and implementation of the proposed libraries will be an iterative process.
Starting from a mock-up of what the proof of the matrix semiring theorem could look like in Agda with the proposed libraries, I will aim to design and implement the first iteration of finite sets, basic linear algebra and big operators during Lent term, leaving some time at the end to critique my first attempt thoroughly and design the second iteration with the lessons learned from the first one in mind.

\subsection{Easter vacation and Easter term}

I will use the Easter vacation and the beginning of Easter term to implement the second iteration of the libraries. If time permits, I will attempt to prove some theorems in the area of shortest path problems.

\subsection{Dissertation}

In the spirit of Simon Peyton-Jones' slogan \enquote{writing a paper guides your research}, I aim to write my dissertation as the project proceeds in the hope that writing about what I am doing will help me understand it better. The last four weeks before the dissertation is due will be reserved for polishing up the dissertation and the code.

\printbibliography

\end{document}
