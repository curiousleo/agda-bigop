\documentclass[a4paper]{scrartcl}
% generated by Docutils <http://docutils.sourceforge.net/>
% rubber: set program xelatex
\usepackage{xltxtra} % loads fixltx2e, metalogo, xunicode, fontspec
\usepackage{microtype}      % typography pedantry
\usepackage[british]{babel}
% \defaultfontfeatures{Scale=MatchLowercase}
\usepackage{ifthen}
\usepackage{color}
\usepackage{amsthm}
\usepackage{thmtools}
\usepackage{paralist}
\usepackage{nameref}
\usepackage{cleveref}
\usepackage{csquotes}
\usepackage{fancyvrb}
\usepackage[autocite=footnote,citestyle=authoryear-comp,bibstyle=authoryear,
            dashed=false,isbn=false,doi=false,backend=biber]{biblatex}
\usepackage[bookmarks,hidelinks]{hyperref}

\setcounter{secnumdepth}{0}

%%% Custom LaTeX preamble
% Linux Libertine (free, wide coverage, not only for Linux)
\setmainfont{Linux Libertine O}
\setsansfont{Linux Biolinum O}
\setmonofont[HyphenChar=None,Scale=MatchLowercase]{DejaVu Sans Mono}

%%% User specified packages and stylesheets
\DefineVerbatimEnvironment%
   {code}{Verbatim}
   {} % options

% hyperlinks
\urlstyle{same} % normal text font (alternatives: tt, rm, sf)
\hypersetup{%
  pdftitle={Routing with composure},
  pdfauthor={Leonhard Markert (lm510), Emmanuel College}
}

% KOMA-script styles
%\addtokomafont{disposition}{\rmfamily}

% bibliography
\addbibresource{Bibliography.bib}

%%% Title Data
\title{\phantomsection%
  Routing with composure%
  \label{generalised-big-operators-in-agda}}
\author{}
\date{}

%%% Body
\begin{document}
\maketitle

% Current research into routing protocols seeks to formalise the assumptions made by routing algorithms in terms of the underlying algebraic structures, most notably variations and combinations of semirings.
% 
% This has lead to some surprising results -- for example, Dijkstra's algorithm requires much less structure from the underlying algebraic entity than previously thought.
% 
% The purpose of this project would be to prove one or more variants of the distributed Bellman-Ford algorithm or the Dijkstra algorithm correct while taking care to make as few assumptions as possible about the underlying algebraic structure. This will require me to:
% 
% - Translate the algebraic structures used in the analysis of routing protocols into Agda types and data structures. Some standard entities from abstract algebra already exist in Agda's standard library, others would have to be newly implemented;
% - Find minimal sets of assumptions that make the routing algorithms correct; and
% - Prove the correctness of the routing algorithms based on these assumptions in Agda.


\begin{abstract}
Dijkstra's algorithm in its most common form computes globally optimal paths with respect to some metric, which must satisfy certain laws. Recently, it has been shown that Dijkstra's algorithm computes \emph{locally} optimal paths even when the underlying metric is not distributive. A proof of correctness not assuming distributivity has been formalised in the theorem prover Coq, but it is complex and hard to understand for someone not intimately familiar with Coq. I will reconstruct this proof in Agda, a dependent type theory, aiming to simplify it and make it easier to read. This will require me to develop several libraries for the proof to build upon. These libraries may be of interest for the wider Agda community, so they will be designed with reusability in mind.
\end{abstract}

\section{Introduction%
  \label{introduction}%
}

\section{Workplan%
  \label{workplan}%
}

\printbibliography

\end{document}
